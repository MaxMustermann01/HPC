\documentclass[oneside, a4paper, DIV=10]{scrartcl}


% PACKAGES
\usepackage[english]{babel}

% TITLE
\title{Exercise Sheet VII}
\author{G\"unther Schindler, Klaus Naumann \& Christoph Klein}

% DOCUMENT
\begin{document}
\maketitle

% PART 1
%%%%%%%%

\section*{Reviews}
\begin{itemize}
    % first review
    \item 
    The paper 'Roofline: An Insightful Visual Performance Model for Multicore
    Architectures' from Williams, Waterman and Patterson published in April 2009
    starts with outlining the importance and usefulness of a model, which provides
    performance guidelines for computational kernels on various computer
    architectures.

    They assert that memory bandwidth and the processor's maximal possible
    calculations per time are the main limiting performance factors. They
    visualize these performance limits in a coordinate system, with
    operations per second on the $y$- and operational intensity,  which is
    the amount of operations divided by the size of the fetched data
    (e.g. from memory to cache or from cache to processor), on the 
    $x$-axis.

    If a programmer determines the operational intensity of a kernel he can
    estimate the limiting performance factor in the described coordinate
    system. The advantage for the programmer is that he knows how to optimize
    efficiently his code.

    To our mind the Roofline-Model provides a handy guideline for how to
    determining bottlenecks and optimizing efficiently on various 
    computer architectures.
    % second review
    \item
\end{itemize}


\end{document}
